\documentclass[12pt, a4paper]{article}

\usepackage{amsmath}
\usepackage{amssymb}
\usepackage{graphicx}
\usepackage{listings}
\usepackage{color}
\usepackage[section]{placeins}
\usepackage{paralist}
\usepackage{fullpage}
\usepackage{glossaries}

\usepackage{caption}
\usepackage{subcaption}

\usepackage{helvet}
\renewcommand{\familydefault}{\sfdefault}

\newacronym{GSM}{GSM}{Global System for Mobile Communications}
\newacronym{NSA}{NSA}{National Security Agency}
\newacronym{SELinux}{SELinux}{Security-Enhanced Linux}
\newacronym{SIM}{SIM}{Subscriber Identity Module}

\newcommand*{\titleGM}{\begingroup
\hbox{ 
\rule{1pt}{\textheight} 
\hspace*{0.02\textwidth} 
\parbox[b]{0.97\textwidth}{ 

{\noindent\Huge\bfseries Mobile Devices have made us the most monitored individuals in history}\\[2\baselineskip] % Title
{\large \textit{SEM2220 - Assignment 4}}\\[4\baselineskip] % Tagline or further description
{\Large \textsc{090932464}} % Author name

\vspace{0.5\textheight} 
}}
\endgroup}


\begin{document}
\titleGM 
\tableofcontents
\newpage

\section{Introduction}

There has been a lot of news in the press in the last year about surveillance
programs run by many countries security agencies, particularly focused on the
collection of phone records by the \gls{NSA}\cite{greenwald2013nsa}.

It comes as no surprise that this has brought about the question: 

\begin{quote}
``Have mobile devices made us the most monitored individuals in history?''
\end{quote}

To fully understand the depth of this question, one must consider the technical
methods which are employed to monitor mobile devices and how these methods
differ across different mobile platforms. In hand with this, one must
investigate the methods for preventing unwanted monitoring on mobile devices
and how effective these methods really are.

Finally there are many legal and professional issues which govern the
collection of data from mobile devices; but there are also a host of ethical
and social issues which legal and professional element cannot, necessarily,
cover.

Of course monitoring of data, especially from mobile devices, doesn't have to
be used for nefarious purposes. Applications commonly use data mining 
techniques to provide a better experience for their users and cloud services 
are so popular that imagining a world where all a users data is stored solely 
on a single device is almost impossible.

\newpage
\section{Technical Methods for Monitoring Mobile Devices}

Because of the structure of the \gls{GSM}, any mobile device which accesses
the mobile network can be tracked to, at worst, the nearest cell tower. In 
fact the \gls{NSA} is known to use this data to track targets and identify 
possible accomplices\cite{classified2012co-travel}. There are also other
stories of UK public organisations performing similar
actions\cite{gallager2011met}\cite{hopkins2013gchq}.

Though this is a worrying concept, there is some important information that the
U.S. has given in \cite{clapper2013facts} which denies that any information
collected cannot be used to target any individual without a prior specific and
documented reason to do so. In the case of monitoring of non-U.S. citizens
there must also be a agreement with the country of their citizenship.

It is the monitoring which the users agrees to have performed upon them that is
perhaps more worrying.

Mobile devices, especially tablets and smart phones, have a range of sensors
which third-party applications may be able to access. These sensors typically
include a camera, microphone, WiFi antenna and GPS system. As technology
progresses the accuracy of these sensors is increasing and other sensors such
as accelerometers are becoming more prevalent.

It could, therefore, be easy for an attacker to write a malicious application
which simply monitored the GPS location of a person or recorded their
conversations through the microphone. Or even use any of these systems combined
to gather a lot of sensitive data.

There have been cases which have been formally
investigated\cite{ftc2013flashlight} in which legitimate applications collect
data from users on the side, which are then sold to other companies for
advertisement revenue.

However, a lot of applications have legitimate uses for these sensors. Most
applications require network access to load data from external servers. More
specific applications, which perform tasks like route tracking, will require
the use of more than one of these sensors.

\newpage
\section{Protection Against Unwanted Monitoring}

Unfortunately, due to the structure of the \gls{GSM} network, as long as a
mobile device is on and allowed to connect to the network it can be tracked.
With the addition of a \gls{SIM} card it can even be tied to a specific
individual. Under \cite{directive2006/24/EC}

Modern mobile platforms typically have some sort of security system and ways in
which applications can be given access to secure elements of the mobile device.
Most of these platforms sandbox third-party application allowing the kernel to
control the access to the various facilities of the device.

\subsection{Android Security Facilities}

\cite{google2013security} describes the methods that the Android system uses to
secure the environment. The main advantage Android gains in this area is that
uses \gls{SELinux}, which confines the access of programs based on different
policies.

Android adds many features on top of this, the most applicable of these to the
question of monitoring is the permission model. For a third party application
to access protected APIs such at GPS positioning or network connections they
must implicitly define which of these elements they require in the manifest
file of the application.

This information is displayed to the user when they first install the
application and when the application updates to add new access to secure APIs.

There are also certain APIs which are not available to third-party 
applications, but which may be used by pre-installed applications if they are
signed as part of the OS.

Because applications are sandboxed from one another it should be an impossible
task to gain confidential information from another application, unless it
provides it somehow.

\subsubsection{Additional Security Features Provided by the Google Play Store}

Obviously, telling legitimate use of secure API elements from illegitimate use
is not an easy task, especially for automated systems.

In the mobile ecosystem, the typical method of performing this is to have a
trusted place, where developers can distribute their applications. In the case
of Android this is usually the Google Play store, however other stores such as
the Amazon Appstore do also exist.

The maintainers of such places can then regulate the applications accepted,
monitor them for any malicious activities and remove them where needed. 
Google even added a service in 2012 which scanned the Play store for
applications with a malicious intent by running them on a range of emulated
services\cite{lockheimer2012android}.

\nocite{lockheimer2012android}

\subsection{iOS Security Facilities}
\cite{apple2012security} defines the methods that the iOS system uses to secure
the environment. Similar to Android iOS third-party applications are sandboxed 
from each other, although a lot of this is done by the iOS runtime. All
applications have their own file spaces which can only be accessed through the
iOS defined API.

Access to secure API elements is defined through entitlements; signed key-value
pairs specific to applications. However, it should be noted that these
entitlements are not required to access elements such as the GPS or microphone
and are more typically used by system applications and daemons to perform tasks
that would typically require root access.

The iOS ecosystem is heavily reliant on the Apple App
Store\cite{apple2013appstore} to provide security against the malicious
collection of data from iOS applications.

%\subsection{Windows 8 Phone Security Features}



\newpage
\section{Implications of Mobile-based Monitoring}

There are several laws which govern the legalities of computer software and the
storage of personal data. The notable laws which apply to the monitoring of
data gathered by mobile devices are:

\begin{itemize}
\item Data Protection Act\cite{dpa1998}
\item Computer Misuse Act\cite{cma1990}
\end{itemize}

Another, perhaps overlooked, law which could relate in these situations is the
Contracts law and Licenses.


%TODO place this nicely
% CMA
The Computer Misuse Act does not affect the use of mobile devices for monitoring
purposes as much as one would suspect. The main issue comes that section 1 of
this act defines misuse as ``the unauthorised access to computer
material''\cite{cma1990}. Sections 2 and 3 also define other forms of misuse, 
but as monitoring would fall under the access of computer material, they do not
apply.

The problem comes that the access could be implied simply by the user installing
the application. Most applications will also have terms and conditions or will
be distributed in such a way that the user is accepting the access to
information providing by sensors before they install the application (e.g. the
Android permission system).

Obviously, if an application could hide the fact it was collecting data from a
permission system and does not disclose this use in terms and conditions, then
an argument could be made that it is breaking this act.
% CMA


\newpage
\bibliographystyle{IEEEtran}
\bibliography{citations}

\end{document}
